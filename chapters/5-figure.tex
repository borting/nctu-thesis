\chapter{Figures and Tables}
\label{chapter:fig}

關於怎麼使用圖和表格, 網路上都可以找到許多介紹. Google it.
(其實是我累了, 不想寫了 XD)

\section{Figures}

這裡介紹如何載入一張圖片, 解釋請看 .tex 檔的註解吧!
此外, 我個人習慣是把所有的圖檔都放在一個資料夾下, 像是 \textit{figures/}

\begin{figure}[b]
  \centering
  % 圖片的高度與寬度, height 設為 ! 代表由寬度大小等比例縮放
  \includegraphics[height=!,width=0.4\linewidth,keepaspectratio=true]%
  % 圖片的位置
  {figures/nctu_logo}
  % [] 放的是顯示在 list of figure 的文字
  % {} 放的是顯示在圖下方的文字
  \caption[NCTU logo]{{\footnotesize The history of NCTU days back to 1896 ...}}
  \label{fig:nctu_logo}
\end{figure}

\textbf{注意:} 在呼叫圖的標籤的時候, 請寫 \textbf{Figure$\sim$\textbackslash ref\{label\}}.
(請參考 .tex 檔看我是如何載入 Figure~\ref{fig:nctu_logo} 的吧).

\subsection{Figure indexing}
\label{subsec:fig_index}

目前圖、表、公式的編號是用流水號 (1, 2, 3, ...),如果需要修改成 book class 類似的章節號 (1.1, 1.2, 2.1, ...),可以到 main.tex 將 counterwithout{} 註解掉。

\section{Tables}

我個人的習慣是把表格的內容放到另一個 .tex 內, 再把這些 .tex 檔放到另一個資料夾下 (e.g. \textit{tables/}), 讓本文看起來不會那麼亂.
請參考 Table~\ref{table:clsoptions} 裡面的註解 (檔案位置 \textit{tables/table-classopt.tex}), 看看要怎寫一個簡單的 table 吧.
